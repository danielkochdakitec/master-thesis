\chapter{Grundlagen}

Dieses Kapitel stellt die Grundlagen der Arbeit dar. Diese lassen sich in die Abschnitte Begriffe sowie \ac{AWS} Amplify und Firebase aufteilen. Der erste Abschnitt definiert die Begriffe \ac{FaaS} und \ac{BaaS} und grenzt diese von den weiteren Modellen ab. Darauf folgt die Vorstellung der \acl{BaaS} Tools \ac{AWS} Amplify und Firebase.

\section{Begriffe}

Einer der Vorreiter des Cloud Computing ist John McCarthy, der in einer Rede am Massachusetts Institute of Technology (MIT) 1961 den Begriff des Utility Computings prägte. In dieser Rede ging er davon aus, dass in der Zukunft Rechenleistung ebenso wie das Telefonsystem als öffentliche Versorgung organisiert sind. Vergleichbar soll diese Versorgung dann mit Wasser, Strom und Gas sein. Daraus haben sich bis heute einige große Cloud-Anbieter wie \ac{AWS}, Google Cloud Platform oder Microsoft Azure entwickelt. Deren Geschäftsmodell begann damit, dass sie ihre überschüssige Rechenleistung als öffentlichen Dienst angeboten haben. Dies wird aber mehr und mehr zum Kerngeschäft dieser Unternehmen. \autocite{buyya2013mastering}

Allgemein lässt sich die Cloud in verschiedene Service-Modelle wie \ac{IaaS}, \ac{PaaS}, \ac{FaaS} oder \ac{SaaS} unterteilen. Es gibt noch weitere Modelle wie \ac{BaaS}, \ac{MBaaS}, die eine Kombination aus mehreren Service-Modellen sind. Die Definition, Abgrenzung und Einordnung dieser Modelle wird in verschiedener Literatur \autocite{jiang2020overview} \autocite{kumar2019serverless} \autocite{dahunsi2021commercial} behandelt.

  \subsection{Function-as-a-service}
  \subsection{Backend-as-a-service}
  - was ist das genau und wie funktioniert es
  - was sind gängige beispiele dafür
  \subsection{Infrastructure-as-a-service}
- Infrastructure as Code (?)

  \subsection{Platform-as-a-service}

\section{AWS Amplify}

\autocite{dahunsi2021commercial}
\autocite{amplifyDocs}
\autocite{lysakov2021security}
\autocite{mathew2014overview}
\autocite{beach2014aws}

  - Amplify
  - Amplify CLI
  - Custom Plugins
  - Wie sieht die Infrastrutkur aus, wie der code, wie das frontend?
  - CloudFormation
  - AppSync
  - Lambda
  - S3
  - Cognito
  - MediaConvert
  - Amazon EventBridge
  - DynamoDB
  - CloudFront (weil kurz erwähnt)

\section{Firebase}

\autocite{moroney2017definitive}
\autocite{firebaseDocs}
\autocite{tanna2018serverless}

- Weitere BaaS tools

Allgemeines:
  - Anzahl der Nutzer
  - Zufriedenheit / Akzeptanz
  - Zeit im Markt
  - Sonstige Fakten
  - GitHub Stars