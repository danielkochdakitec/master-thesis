\chapter{Vergleich}

todo einleitungstext

ggf. generelle Vergleichspunkte (Anzahl der Nutzer, Zufriedenheit, Akzeptant, etc., Zeit im Markt)

\section{Cloud-Kosten}


- Übersicht über Kostenpunkte
  - Was gibt es für gemeinsamkeiten (Vergleich)
  - Was ist bei AWS kostenlos
  - Was ist bei Firebase kostenlos
  - Free tier????
  - Beispiel-Schema der Berechnung
  - Endübersicht

- Preise / Cloud-Kosten
  - Break-Even Point grafik für preise
  - Preismodell



  Projektkosten:
  https://calculator.aws/#/estimate?id=b76bdc9aa6582ee140d631ee394dc6b0e52048f7

  Frankfurt, monatliche Kosten, exklusive Free-Tier, da manche Angebote nur 1 Jahr gültig sind

  AWS

AWS Lambda (x86)

  Paramater:
  - Number of requests: 10.000
  - Duration of each request: 300ms
  - Amount of memory allocated: 256MB

  Berechnung:
  - Compute Charges= Anzahl request * Durchschnittliche Request-Zeit * Durchschnittlich Benutzer Speicher * 0.0000166667 USD (1*sec*GB*USD/GBs)
  - Request Charges = Anzahl request * 0.0000002 USD

  Amazon EventBridge

  Parameter:
  - Payload Size: 10KB
  - Custom Events: 1 Million per Month

  Berechnung:
  - Event Costs: 1 Mio Custom Events x 0.000001 USD

  Amazon Cognito
  - Without advanced security features: free
  - We don’t need advanced security features here

  S3

  Parameter
  - S3 Standard Storage in GB
  - PUT, COPY, POST, LIST Requests
  - GET, SELECT Requests
  - Inbound Data Transfer: Free
  - Outbound Data Transfer: GB

  Berechnung:
  - S3 Standard Storage Costs: S3 Standard Storage in GB * 0.0245000000 USD
  - S3 Standard PUT request costs: Number of PUT requests * 0.0000054 USD per request
  - S3 Standard GET request costs: Number of GET requests * 0.00000043 USD per request
  - Outbound Data Transfer:
      - 10240 GB x 0.09 USD per GB = 921.60 USD (next 10TB)
      - 40960 GB x 0.085 USD per GB = 3481.60 USD (next 40TB)
      - 102400 GB x 0.07 USD per GB = 7168.00 USD (next 100TB)
      - 999846400 GB x 0.05 USD per GB = 49992320.00 USD (larger than 150TB)

  DynamoDB

  Parameter:
  - Data storage size in GB
  - Average item size
  - Writes
      - Standard writes vs Transactional writes: 100 / 0
      - Number of writes: 1000
  - Reads
      - Eventually consistent vs Strongly vs Transactional: 100 / 0 / 0
      - Number of reads: 10 million per month
  - On demand backup data storage

  Berechnung:
  - Data storage costs: Data storage size * 0.306 USD
  - Write costs: Number of writes * 0.000001525 USD
  - Read costs: Number of reads x 1 strongly consistent portion x 1 read request units for strongly consistent reads x 25 read request units needed per item = 7,850,000,000.00 read request units for strongly consistent reads * 0.000000305 USD
  - Backup storage costs: On demand backup data storage x 0.1224 USD

  AWS Amplify

  Parameter:
  - Number of build minutes
  - Data stored per month GB
  - Data served per month GB

  Berechnung:
  - Build minutes cost = Number of build minutes * 0,01 USD / min
  - Data stored cost = Data stored per month GB * 0.023 USD / GB * mon
  - Data served cost = Data served per month GB * 0.15 USD / GB * mon

  AWS Elemental MediaConvert

  Parameter:
  - Tier: basic (Fix), Video-Codec: AVC, Single pass, HD, <= 30 FPS
  - Output usage minutes per month

  Berechnung:
  - Outputs: Output usage minutes per month * 0.017 USD / minute

  AppSync

  Parameter:
  - Query and Data Modification Operations per million
  - Average response size: 3KB

  Berechnung:
  - OperationCosts = Query and Data Modification Operations per million * $4.00 USD
  - Data Transfer Costs = Avg response size * Modifications * $0.09 USD /.GB

















\section{Entwicklungsaufwand}

- Welche Komplnenten kann man verwenden vs muss man selbst coden


- Architektur und Implementierung
  - Auth:
    - Amplify hat Groups in Cognito, Cognito ist umfangreicher
    - Authentication hat keine Gruppen, daher eigene Tabelle und Logik (= umfangreicher)
  - Manuelle Anpassung von Ressourcen: Einige Anpassungen an von Amplify bereitgestellten Ressourcen lassen sich im CloudFormation-Template nicht verändern. Der Build-Prozess überschreibt die Veränderung, so dass folgendes Szenario nicht möglich ist: Im \textit{AuthStack} muss die Identifikation des User-Pools statt die des Identity-Pools genutzt werden, damit die Lambda-Funktion weiß, um welchen Nutzer es sich handelt. Als Workaround muss die Änderung manuell in die Ressource eingetragen werden. Das ist sehr fehleranfällig, da bei einem erneuten Setup des Projektes diese manuelle Änderung immer wieder eingespielt werden muss.
    - abhängig von Amplify-CLI, selbst erstellen fast kaum möglich
 - SAM in Amplify, Emulator gibt es bei Firebase
 - Firebase hat keine einheitliche AppSync Schnittstelle, die direkt auf dynamodb resolved, sondern man muss das selbst schreiben
 - Serverless, CDK, FIrebase ist infraaturcture im code
 - Amplify generell komplexer als Firebase, Setup war schwerer, mehr Fehler, Firebase war einfacher
 - Firebase Storage kennt keine privaten user-Spaces, amplify hat das direkt mit S3 integriert
- Wiederverwendbarkeit von Code
- Entwicklungszeit
- Wartbarkeit
- Flexibilität des Frameworks
- sonstigen Risiken