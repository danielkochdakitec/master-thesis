\chapter{Vergleich}

- Unterschiede in der Implementierung
- Unterschiede in der Architektur

- Preismodell
- Cloud-Kosten
- Wiederverwendbarkeit von Code
- Entwicklungszeit
- Wartbarkeit
- Flexibilität des Frameworks
- sonstigen Risiken

tbd

- Break-Even Point grafik für preise

- Serverless
- CDK

- CDK als Alternative zu AWS Amplify, da es wesentlich flexibler

Risiken und technische Schulden bei Amplify

Manuelle Anpassung von Ressourcen:

Einige Anpassungen an von Amplify bereitgestellten Ressourcen lassen sich im CloudFormation-Template nicht verändern. Der Build-Prozess überschreibt die Veränderung, so dass folgendes Szenario nicht möglich ist: Im \textit{AuthStack} muss die Identifikation des User-Pools statt die des Identity-Pools genutzt werden, damit die Lambda-Funktion weiß, um welchen Nutzer es sich handelt. Als Workaround muss die Änderung manuell in die Ressource eingetragen werden. Das ist sehr fehleranfällig, da bei einem erneuten Setup des Projektes diese manuelle Änderung immer wieder eingespielt werden muss.

abhängig von Amplify-CLI, selbst erstellen fast kaum möglich


- SAM in Amplify
- Emulator gibt es bei Firebase

- Amplify hat Groups in Cognito, Cognito ist umfangreicher
- Authentication hat keine Gruppen, daher eigene Tabelle und Logik (= umfangreicher)

- Firebase hat keine einheitliche AppSync Schnittstelle, die direkt auf dynamodb resolved, sondern man muss das selbst schreiben

Amplify generell komplexer als Firebase