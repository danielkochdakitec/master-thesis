\chapter{AWS Amplify}

Das folgende Kapitel beschreibt die Architektur und die Implementierung mit Amplify.

\section{Architektur}

\subsection{Lösungsstrategie}

Die Architektur folgt dem Aufbau einer Drei"=Schichten"=Architektur:

\begin{itemize}
  \item Präsentationsschicht
    \begin{itemize}
      \item React.js
      \item amplify-js
    \end{itemize}
  \item Logikschicht
    \begin{itemize}
      \item AppSync
      \item Lambda
      \item Cognito
      \item Elemental MediaConvert
      \item EventBridge
      \item IAM
    \end{itemize}
  \item Datenhaltungsschicht
    \begin{itemize}
      \item DynamoDB
      \item S3
      \item Cognito
      \item IAM
      \item AWS Backup
    \end{itemize}
\end{itemize}

Die Cloud-Dienste Cognito und IAM befindet sich zugleich in zwei Schichten, da dort neben der Logik für Authentifizierung und Autorisierung auch die Benutzer persistiert werden.

Durch die ausschließliche Verwendung interner Cloud-Dienste, keiner Eigenentwicklungen für die Benutzerverwaltung und definierter Technologien sind die Rahmenbedingungen für dieses Projekt nicht missachtet, so dass die Vergleichbarkeit weiterhin gegeben ist.

\autoref{Kap2:Teildisziplinen} zeigt erste Lösungsansätze, um die nichtfunktionalen Anforderungen, auch Qualitätsziele genannt, zu erreichen.

\begin{table}[h]
  \caption{Lösungsansatz für nichtfunktionale Anforderungen}
  \label{Kap2:Teildisziplinen}
  \renewcommand{\arraystretch}{1.2}
  \centering
  \sffamily
  \begin{footnotesize}
    \begin{tabularx}{0.9\textwidth}{l l X}
      \toprule
      \textbf{Anforderung} & \textbf{Dienst} & \textbf{Ansatz}\\
      \midrule
        \emph{Verfügbarkeit} & Amplify & Amplify bietet für Frontend und Backend ein Zero-Downtime Deployment. \\
        \emph{Skalierbarkeit} & - & Durch die Serverless-Architektur skalieren Dienste automatisch innerhalb der \ac{AWS}-Cloud.\\
        \emph{Analysierbarkeit} & CloudWatch & Durch den Einsatz von CloudWatch können Anfragen für jeden Cloud-Dienst nachvollzogen werden.\\
        \emph{Interoperabilität} & AppSync & Durch den Einsatz von AppSync ist mit GraphQL eine einheitliche Schnittstelle sichergestellt.\\
        \emph{Backups} & AWS Backup & Durch den Einsatz von AWS Backup werden automatisiert Backups aufgesetzt.\\
        \emph{Wiederherstellbarkeit} & AWS Backup &  Das von AWS Backup erstellte Backup für die Datenbank DynamoDB kann manuell eingespielt werden.\\
      \bottomrule
    \end{tabularx}
  \end{footnotesize}
  \rmfamily
\end{table}

\subsection{Bausteinsicht}

tbd grafik + beschreibung

\subsection{Laufzeitsicht}

\subsubsection{Lesen, Aktualisieren und Löschen eines Videos}

Der folgende Prozess beschreibt, wie Videos von Nutzern gelesen, aktualisiert oder gelöscht werden.

\begin{figure}
  \centering
  \includegraphics[width=0.75\columnwidth]{5_aws-amplify/laufzeitsicht_1.pdf}
  \caption{Laufzeitsicht - Lesen, Aktualisieren und Löschen eines Videos}
  \label{Amplify:laufzeitsicht1}
\end{figure}

\subsubsection{Erstellen eines Videos}

Der nächste Prozess beschreibt den Ablauf, wie ein Video von einem Nutzer erstellt wird.

\begin{figure}
  \centering
  \includegraphics[width=1\columnwidth]{5_aws-amplify/laufzeitsicht_2.pdf}
  \caption{Laufzeitsicht - Erstellen eines Videos}
  \label{Amplify:laufzeitsicht2}
\end{figure}

\subsubsection{Erstellen von signierten Links}

Dieser Prozess beschreibt, wie Nutzer signierte temporäre Links erstellen können, um das Video anzuzeigen.

\begin{figure}
  \centering
  \includegraphics[width=1\columnwidth]{5_aws-amplify/laufzeitsicht_3.pdf}
  \caption{Laufzeitsicht - Erstellen von Links}
  \label{Amplify:laufzeitsicht3}
\end{figure}

\subsection{Verteilungssicht}

\begin{figure}
  \centering
  \includegraphics[width=1\columnwidth]{5_aws-amplify/verteilungssicht.pdf}
  \caption{Verteilungssicht}
  \label{Amplify:verteilungssicht}
\end{figure}

Wo kommt hin, welcher Stack was beinhaltet?
Es gibt keine Substacks, aber weitere parentstacks - das ist vermutlich für die Debuggbarkeit.

\subsection{Querschnittliche Konzepte und Muster}

- Einbindung über \ac{AWS} Amplify:
  - AWS Elemental MediaConvert als extra Plugin
  - Anpassung an CF-Templates
- Logging (CloudWatch)
- Permissions (IAM)
- AWS Backup

\subsection{Entwurfsentscheidungen}

\subsection{Risiken und technische Schulden}

- Auth: Anpassung, dass statt Identity ID die Pool User Id genutzt wird geht über https://github.com/aws-amplify/amplify-js/issues/54. Leider lässt sich das nicht persistieren, so dass man das bei jedem branch/Projekt neu eintragen muss. Wenn man es im CF-Template ändert, wird die Änderung beim Push rückgängig gemacht. Das ist nicht sehr flexibel. Ggf. Lösung finden und schauen, ob es doch geht.

- CDK als Alternative zu AWS Amplify, da es wesentlich flexibler

\section{Implementierung}