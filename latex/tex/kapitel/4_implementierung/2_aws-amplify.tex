\section{AWS Amplify}

Der erste Abschnitt stellt die Implementierung mittels \ac{AWS} Amplify vor.

\subsection{Lösungsstrategie}

Die Architektur folgt dem Aufbau einer Drei"=Schichten"=Architektur:

\begin{itemize}
  \item Präsentationsschicht
    \begin{itemize}
      \item React.js
      \item amplify-js
    \end{itemize}
  \item Logikschicht
    \begin{itemize}
      \item AppSync
      \item Lambda
      \item Cognito
      \item MediaConvert
    \end{itemize}
  \item Datenhaltungsschicht
    \begin{itemize}
      \item DynamoDB
      \item S3
      \item Cognito
    \end{itemize}
\end{itemize}

Der Cloud-Dienst Cognito befindet sich in zwei Schichten, da dort neben der Logik für Authentifizierung und der Autorisierung auch die Benutzer persistiert werden. Durch die ausschließliche Verwendung interner Cloud-Dienste, keiner Eigenentwicklungen für die Benutzerverwaltung und definierter Technologien sind die Rahmenbedingungen für dieses Projekt nicht missachtet, so dass die Vergleichbarkeit gewährt bleibt.

\autoref{Kap2:Teildisziplinen} zeigt erste Lösungsansätze, um die nichtfunktionalen Anforderungen, auch Qualitätsziele genannt, zu erreichen.

\begin{table}[h]
  \caption{Lösungsansatz für nichtfunktionale Anforderungen}
  \label{Kap2:Teildisziplinen}
  \renewcommand{\arraystretch}{1.2}
  \centering
  \sffamily
  \begin{footnotesize}
    \begin{tabularx}{0.9\textwidth}{l X}
      \toprule
      \textbf{Anforderung} & \textbf{Ansatz}\\
      \midrule
        \emph{Verfügbarkeit} & Amplify bietet für Frontend und Backend ein Zero-Downtime Deployment. \\
        \emph{Skalierbarkeit} & Durch die Serverless-Architektur skalieren Services automatisch innerhalb der \ac{AWS}-Cloud.\\
        \emph{Analysierbarkeit} & Durch den Einsatz von CloudWatch für jeden Service können Anfragen nachvollzogen werden.\\
        \emph{Interoperabilität} & Durch den Einsatz von AppSync ist mit GraphQL eine einheitliche Schnittstelle sichergestellt.\\
        \emph{Backups} & Durch den Einsatz von AWS Backup werden automatisiert Backups aufgesetzt.\\
        \emph{Wiederherstellbarkeit} & Das von AWS Backup erstellte Backup für die Datenbank DynamoDB kann manuell eingespielt werden.\\
      \bottomrule
    \end{tabularx}
  \end{footnotesize}
  \rmfamily
\end{table}

\subsection{Bausteinsicht}

tbd grafik + beschreibung

\subsection{Laufzeitsicht}

tbd grafik + beschreibung

\subsection{Verteilungssicht}

tbd grafik + beschreibung

\subsection{Querschnittliche Konzepte und Muster}

- Einbindung über \ac{AWS} Amplify:
  - MediaConvert als extra Plugin
  - Anpassung an CF-Templates
- Logging (CloudWatch)
- Permissions (IAM)
- AWS Backup

\subsection{Entwurfsentscheidungen}

\subsection{Risiken und technische Schulden}

- Auth: Anpassung, dass statt Identity ID die Pool User Id genutzt wird geht über https://github.com/aws-amplify/amplify-js/issues/54. Leider lässt sich das nicht persistieren, so dass man das bei jedem branch/Projekt neu eintragen muss. Wenn man es im CF-Template ändert, wird die Änderung beim Push rückgängig gemacht. Das ist nicht sehr flexibel. Ggf. Lösung finden und schauen, ob es doch geht.

- CDK als Alternative zu Amplify, da es wesentlich flexibler