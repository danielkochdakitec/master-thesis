\chapter{Verwandte Arbeiten}

Dieses Kapitel stellt weitere Arbeiten dar, die sich mit den Backend-as-a-service Plattformen \ac{AWS} Amplify und Firebase beschäftigt haben.

Obwohl \ac{AWS} Amplify verglichen mit Firebase relativ neu am Markt ist, gibt es einige wissenschaftliche Arbeiten, die sich damit auseinandergesetzt haben. Die Beispiele zeigen wie vielfältig die Plattformen genutzt werden können.

Die mobile App \textit{Handy Doctor} \autocite{st2021handy} beschäftigt sich mit der Aufgabe, Arztbesuche während der Corona-Pandemie zu vermeiden. Dabei kommen \ac{AWS} Amplify gemeinsam mit Amazon Polly und Amazon Translate zum Einsatz, um geeignete Möglichkeiten zum Chatten und Telefonieren mit einem Arzt herzustellen.

Eine weitere App bietet ein Template-System für Online-Umfragen mit Hilfe von Gamification-Elementen \autocite{kuwamura2021application}. Dort kommt AWS Amplify gemeinsam mit TypeScript und React zum Einsatz.

Eine dritte Plattform soll Brücken überwachen \autocite{naraharisetty2021cloud}. Dabei hilft \ac{AWS} Amplify, um möglichst schnell Frontend-Komponenten aufzubauen und zu verteilen.

Dadurch, dass Firebase schon länger am Markt ist, gibt es einige Arbeiten mehr über Software, die mit Firebase erstellt wurde.

Die App \textit{Project C} \autocite{rahman2021project} beschäftigt sich mit dem Tracking von Corona-Positiven sowie die Benachrichtigung derer Personen, die in der Nähe waren. Zur Umsetzung wurde neben Firebase die Google Nearby Messages API sowie die Android SDK genutzt. Die App ist damit nur für Android umgesetzt.

Auch die nächste App beschäftigt sich mit einem medizinischen Thema. Bei der App \textit{FinDoctor} \autocite{rahmi2017findoctor} geht es um die Kommunikation und das Auffinden eines Arztes. Neben Firebase wird auch die Google Maps API für die Navigation zum Arzt genutzt.

Auch weitere Arbeiten beschäftigen sich mit Lösungen im Bereich IOT oder Kommunikation \autocite{li2018justiot} \autocite{sharma2019firebase} \autocite{bhadoria2020chatapp} \autocite{khawas2018application}.

Zusammenfassend soll auch diese Arbeit sich zum Ziel setzen, ebenfalls eine Software zu planen und prototypisch umzusetzen. Dabei sollen beide Backend-as-a-service Plattformen \ac{AWS} Amplify und Firebase auf die Wirtschaftlichkeit verglichen werden.