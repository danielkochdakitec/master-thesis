\section{Motivation}

Der Begriff des "`Software Engineering"' wurde erstmal 1968 auf einer Nato-Konferenz definiert \autocite{naur1969software}. Verglichen mit anderen Ingenieursdisiziplinen ist die Softwareentwicklung damit eine sehr junge Disziplin. Vorgehensweisen der Softwareentwicklung unterliegen dementsprechend einem stetigen Wandel.

Während Entwickler zu Beginn vor allem monolithische Architekturen entworfen haben, sind innerhalb kürzester Zeit weitere Architekturformen hinzugekommen. Neben der serviceorientierten Architektur spielen heute vor allem Architekturen basierend auf Microservices eine immer größer werdende Rolle. Auch das Deployment von Software ändert sich stetig. Während Entwickler zu Beginn Software auf einzelnen Servern ausgeliefert haben, ging der Trend über virtuelle Maschinen hin zu Container-Technologien wie Docker oder der Auslieferung von Software über Cloud-Dienste wie \ac{AWS} oder Google Cloud.

Eines weiterer Trend seit der Nutzung der Cloud ist die Verwendung von Backend-as-a-service Plattformen wie \ac{AWS} Amplify oder Firebase. Mit solchen Plattformen möchten Cloud-Anbieter es schaffen, durch hohe Wiederverwendbarkeit die Entwicklung von Web-Anwendungen massiv zu vereinfachen. Dadurch sollen Unternehmen die Ausgaben für die Entwicklung von Web-Anwendungen reduzieren und gleichzeitig nur für die Rechenleistung zahlen, die sie auch wirklich nutzen \autocite{villamizar2017cost}.