\section{Motivation}

Der Begriff "`Software Engineering"' wurde erstmals 1968 auf einer Nato-Konferenz definiert \autocite{naur1969software}. Verglichen mit anderen Ingenieursdisiziplinen ist die Softwareentwicklung eine recht neue Disziplin. Vorgehensweisen der Softwareentwicklung unterliegen dementsprechend einem stetigen Wandel und brauchen Zeit, sich entsprechend zu etablieren.

Während Entwickler zu Beginn vor allem monolithische Architekturen entworfen haben, sind innerhalb kürzester Zeit weitere Architekturformen hinzugekommen. Neben der serviceorientierten Architektur spielen heute vor allem Architekturen basierend auf Microservices eine immer größer werdende Rolle. Auch das Deployment von Software ändert sich stetig. Während Entwickler zu Beginn Software auf einzelnen Servern ausgeliefert haben, ging der Trend über virtuelle Maschinen hin zu Container-Technologien wie Docker oder der Auslieferung von Software über Cloud-Dienste.

Einer der Vorreiter des Cloud Computings ist John McCarthy, der in einer Rede am Massachusetts Institute of Technology (MIT) 1961 den Begriff des Utility Computings prägte. In dieser Rede ging er davon aus, dass in der Zukunft Rechenleistung ebenso wie das Telefonsystem als öffentliche Versorgung organisiert sind. Vergleichbar soll diese Versorgung dann mit Wasser, Strom und Gas sein. Entsprechend haben sich bis heute einige große Cloud-Anbieter wie \ac{AWS}, Google Cloud Platform oder Microsoft Azure entwickelt. Deren Geschäftsmodell begann damit, dass sie ihre überschüssige Rechenleistung als öffentlichen Dienst angeboten haben. Dies wird aber mehr und mehr zum Kerngeschäft dieser Unternehmen. \autocite{buyya2013mastering}

Außerdem ist damit ein weiterer Trend aufgekommen, die Verwendung von \ac{BaaS}-Plattformen wie \ac{AWS} Amplify oder Firebase. Mit solchen Plattformen möchten Cloud-Anbieter es schaffen, durch hohe Wiederverwendbarkeit die Entwicklung von Web-Anwendungen massiv zu vereinfachen. Dadurch sollen Unternehmen Ausgaben für die Entwicklung von Web-Anwendungen einsparen und gleichzeitig nur für die Rechenleistung zahlen, die sie auch wirklich nutzen \autocite{villamizar2017cost}.