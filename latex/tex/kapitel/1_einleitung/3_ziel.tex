\section{Zielsetzung und Methodik}

Das Ziel der Arbeit ist es, AWS Amplify und Firebase zu vergleichen. Dazu werden die folgenden Anforderungen für ein Video-Portal definiert:

<todo>: warum macht ein Vergleich Sinn? Was wird genau verglichen? Was sind die Metriken, nach denen man das vergleicht?

Die funktionalen Anforderungen beschreiben, was die Video"=Plattform leisten muss:
\begin{description}
   \item[F1 - Registrierung] Das System muss dem Benutzer die Möglichkeit geben, sich zu registrieren. Dies erfolgt unter Angabe der Attribute E"=Mail, Telefonnummer und Passwort. \label{F1}
   \item[F2 - Login] Das System muss dem Benutzer die Möglichkeit geben, sich mit seiner E"=Mail Adresse und seinem Passwort einzuloggen. \label{F2}
   \item[F3 - Passwort vergessen] Das System muss dem Benutzer die Möglichkeit geben, sein Passwort durch eine Identitätsprüfung zu ändern. \label{F3}
   \item[F4 - Video hochladen] Das System muss dem Benutzer die Möglichkeit geben, ein neues Video zu erstellen. Dabei gibt der Benutzer die Attribute Titel, Beschreibung sowie die Videodatei im Format mp4 an. Das System muss das Video daraufhin automatisch mit einem Wasserzeichen versehen. Sobald dieser Prozess abgeschlossen ist, muss das System das Video freischalten, damit es beim Benutzer angezeigt wird. Ist der Prozess noch nicht abgeschlossen, soll dem Benutzer kein Video angezeigt werden. \label{F4}
   \item[F5 - Videos auflisten] Das System muss dem Benutzer die Möglichkeit geben, alle Videos aufzulisten. Die Videos müssen über das Titel"=Feld durchsuchbar sein. Dabei ist eine exakte Suche ausreichend. \label{F5}
   \item[F6 - Video bearbeiten] Das System muss dem Benutzer die Möglichkeit geben, eigene Videos zu bearbeiten, um den Titel oder die Beschreibung zu ändern. Trägt der Benutzer die Rolle des Administrators, darf dieser auch fremde Videos bearbeiten. \label{F6}
   \item[F7 - Videos löschen] Das System muss dem Benutzer die Möglichkeit geben, selbst hochgeladenen Videos zu löschen. Trägt der Nutzer die Rolle des Administrators, darf dieser auch fremde Videos löschen. \label{F7}
\end{description}

Die nichtfunktionalen Anforderungen beschreiben die Qualitätsmerkmale der zu entwickelnden Video"=Plattform:
\begin{description}
   \item[NF1 - Verfügbarkeit] Das System muss zu 99,9\% pro Jahr verfügbar sein. Ein Deployment darf nicht zu einem Systemausfall führen.\label{NF1}
   \item[NF2 - Skalierbarkeit] Das System muss bei Lastspitzen automatisch skalieren.\label{NF2}
   \item[NF3 - Analysierbarkeit] Das System muss so aufgesetzt werden, dass jeder Backend"=Aufruf durch Log"=Dateien nachvollziehbar ist.\label{NF3}
   \item[NF4 - Interoperabilität] Das System muss gängige Schnittstellen wie REST oder GraphQL anbieten.\label{NF4}
   \item[NF5 - Backups] Das System muss automatisiert Backups für mindestens sieben Tage anlegen.\label{NF6}
   \item[NF6 - Wiederherstellbarkeit] Das System muss im Falle eines Ausfalls oder eines Datenverlusts innerhalb von einer Stunde wiederherstellbar sein.\label{NF7}
\end{description}

Um eine möglichst gute Vergleichbarkeit zwischen \ac{AWS} Amplify und Firebase herzustellen, müssen Rahmenbedingungen für die Video"=Plattform definiert werden:
\begin{description}
   \item[R1 - Keine externen Cloud-Dienste] Das System darf nur Dienste innerhalb der zu entwickelnden Cloud nutzen. Beispielsweise dürfen keine externen Dienste für die Video"=Bearbeitung genutzt werden.
   \item[R2 - Keine Eigenentwicklungen] Das System sollte bei gängigen Problemen weitestgehend bestehende Lösungen der Cloud nutzen, statt Komponenten neu zu entwickeln. Beispielweise sollte eine User"=Authentifizierung nicht von Grund auf neu entwickelt werden. Wenn es keine bestehende Lösung gibt, darf die Software allerdings eine Eigenentwicklung nutzen.
   \item[R3 - Geringe Betriebskosten] Das System muss so konzipiert werden, dass die Betriebskosten der Anwendung minimal gehalten werden.
   \item[R4 - Backend"=Technologie] Das System soll im Backend Node.js, ferner JavaScript oder TypeScript, verwenden.
   \item[R5 - Frontend"=Technologie] Das System soll im Frontend React.js verwenden.
\end{description}


TODO: Nutzungsprofile für die Anwendung definieren, damit der vergleich einfacher ist