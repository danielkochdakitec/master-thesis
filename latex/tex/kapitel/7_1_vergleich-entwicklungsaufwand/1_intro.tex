\chapter{Vergleich des Entwicklungsaufwands}

Das folgende Kapitel vergleicht \ac{AWS} Amplify mit Firebase anhand der entwickelten Plattformen. Dabei liegt der Fokus des Vergleichs auf dem Entwicklungsaufwand.

\section{Emulator}

- SAM in Amplify, Emulator gibt es bei Firebase

\section{Authentication}

- Auth:
  - Welche
  - Amplify hat Groups in Cognito, Cognito ist umfangreicher
  - Authentication hat keine Gruppen, daher eigene Tabelle und Logik (= umfangreicher)

  - Manuelle Anpassung von Ressourcen: Einige Anpassungen an von Amplify bereitgestellten Ressourcen lassen sich im CloudFormation-Template nicht verändern. Der Build-Prozess überschreibt die Veränderung, so dass folgendes Szenario nicht möglich ist: Im \textit{AuthStack} muss die Identifikation des User-Pools statt die des Identity-Pools genutzt werden, damit die Lambda-Funktion weiß, um welchen Nutzer es sich handelt. Als Workaround muss die Änderung manuell in die Ressource eingetragen werden. Das ist sehr fehleranfällig, da bei einem erneuten Setup des Projektes diese manuelle Änderung immer wieder eingespielt werden muss.

\section{GraphQL}

- Firebase hat keine einheitliche AppSync-Schnittstelle, die direkt auf dynamodb resolved, sondern man muss das selbst schreiben

\section{CI/CD Pipeline}

- eigene CI/CD Pipeline für Firebase, Amplify hatte das schon

\section{Einbindung des Transkodierungsdiensts}

- Weitere Dienste einbinden
- abhängig von Amplify-CLI, selbst erstellen fast kaum möglich

\section{Zusammenfassung}

- Amplify generell komplexer als Firebase, Setup war schwerer, mehr Fehler, Firebase war einfacher

<Tabelle mit Kategorien der Entwicklung: wo war was schneller und warum>