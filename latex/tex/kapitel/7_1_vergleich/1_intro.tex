\chapter{Vergleich der entwickelten Software}

Das folgende Kapitel vergleicht \ac{AWS} Amplify mit Firebase. Zunächst werden allgemeine Kriterien der Backend-as-a-service Plattformen verglichen. Darauf folgt der Vergleich von projektspezifischen Metriken mit Fokus auf den Entwicklungsaufwand der jeweiligen Plattform.

\section{Allgemeiner Vergleich}

...
- Frontend Support
- Debugging?

<Tabelle mit Kategorien der Entwicklung: wo war was schneller und warum>

\section{Projektspezifischer Vergleich}

  - Entwicklungsaufwand, Wiederverwendbarkeit

  - SAM in Amplify, Emulator gibt es bei Firebase
  - Auth:
    - Welche
    - Amplify hat Groups in Cognito, Cognito ist umfangreicher
    - Authentication hat keine Gruppen, daher eigene Tabelle und Logik (= umfangreicher)
  - Firebase hat keine einheitliche AppSync-Schnittstelle, die direkt auf dynamodb resolved, sondern man muss das selbst schreiben
  - eigene CI/CD Pipeline für Firebase, Amplify hatte das schon

  - Weitere Dienste einbinden
    - Manuelle Anpassung von Ressourcen: Einige Anpassungen an von Amplify bereitgestellten Ressourcen lassen sich im CloudFormation-Template nicht verändern. Der Build-Prozess überschreibt die Veränderung, so dass folgendes Szenario nicht möglich ist: Im \textit{AuthStack} muss die Identifikation des User-Pools statt die des Identity-Pools genutzt werden, damit die Lambda-Funktion weiß, um welchen Nutzer es sich handelt. Als Workaround muss die Änderung manuell in die Ressource eingetragen werden. Das ist sehr fehleranfällig, da bei einem erneuten Setup des Projektes diese manuelle Änderung immer wieder eingespielt werden muss.
    - abhängig von Amplify-CLI, selbst erstellen fast kaum möglich

    - Amplify generell komplexer als Firebase, Setup war schwerer, mehr Fehler, Firebase war einfacher

    - Wartbarkeit

    <Tabelle mit Kategorien der Entwicklung: wo war was schneller und warum>

\section{Zusammenfassung}

todo