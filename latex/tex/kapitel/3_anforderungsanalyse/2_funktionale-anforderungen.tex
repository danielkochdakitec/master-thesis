\section{Funktionale Anforderungen}

Die funktionalen Anforderungen beschreiben, was die Video"=Plattform leisten muss.

\begin{description}
   \item[F1 - Registrierung] Das System muss dem Benutzer die Möglichkeit geben, sich zu registrieren. Dies erfolgt unter Angabe der Attribute E"=Mail, Telefonnummer und Passwort. \label{F1}
   \item[F2 - Login] Das System muss dem Benutzer die Möglichkeit geben, sich mit seiner E"=Mail Adresse und seinem Passwort einzuloggen. \label{F2}
   \item[F3 - Passwort vergessen] Das System muss dem Benutzer die Möglichkeit geben, sein Passwort durch eine Identitätsprüfung zu ändern. \label{F3}
   \item[F4 - Stammdaten ändern] Das System muss dem Benutzer die Möglichkeit geben, seinen Vornamen und Nachnamen zu ändern. \label{F4}
   \item[F5 - Video hochladen] Das System muss dem Benutzer die Möglichkeit geben, ein neues Video zu erstellen. Dabei gibt der Benutzer die Attribute Titel, Beschreibung sowie die Videodatei im Format mp4 an. Das System muss das Video daraufhin automatisch mit einem Wasserzeichen versehen. Sobald dieser Prozess abgeschlossen ist, muss das System das Video freischalten, damit es beim Benutzer angezeigt wird. Ist der Prozess noch nicht abgeschlossen, soll dem Benutzer kein Video angezeigt werden. \label{F5}
   \item[F6 - Videos auflisten] Das System muss dem Benutzer die Möglichkeit geben, alle Videos aufzulisten, diese nach eigenen Videos oder heute hochgeladenen Videos zu filtern und durchsuchbar anhand des Titel"=Feldes zu machen. \label{F6}
   \item[F7 - Video bearbeiten] Das System muss dem Benutzer die Möglichkeit geben, eigene Videos zu bearbeiten, um den Titel oder die Beschreibung zu ändern. Trägt der Benutzer die Rolle des Administrators, darf dieser auch fremde Videos bearbeiten. \label{F7}
   \item[F8 - Videos löschen] Das System muss dem Benutzer die Möglichkeit geben, selbst hochgeladenen Videos zu löschen. Trägt der Nutzer die Rolle des Administrators, darf dieser auch fremde Videos löschen. \label{F8}
\end{description}