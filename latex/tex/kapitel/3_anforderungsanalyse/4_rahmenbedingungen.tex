\section{Rahmenbedingungen}

Um eine möglichst gute Vergleichbarkeit zwischen AWS Amplify und Firebase herzustellen, müssen Rahmenbedingungen für die Video-Plattform definiert werden.

\begin{description}
   \item[R1 - Keine externen Cloud-Dienste] Das System darf nur Dienste innerhalb der zu entwickelnden Cloud nutzen. Beispielsweise dürfen keine externen Dienste für die Video-Bearbeitung genutzt werden.
   \item[R2 - Keine Eigenentwicklungen] Das System sollte bei gängigen Problemen weitestgehend bestehende Lösungen der Cloud nutzen, statt Komponenten neu zu entwickeln. Beispielweise sollte eine User-Authentifizierung nicht von Grund auf neu entwickelt werden. Wenn es keine bestehende Lösung gibt, darf die Software allerdings eine Eigenentwicklung nutzen.
   \item[R2 - Backend-Technologie] Das System soll im Backend weitestgehend Node.js verwenden.
   \item[R2 - Frontend-Technologie] Das System soll im Frontend React.js verwenden.
\end{description}